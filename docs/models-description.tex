% Some basic notation is defined in 'macros/basic-math-macros'

\documentclass{article}
\usepackage{verbatim}
\usepackage{titlesec}

\def\coursename{bayesDCA supporting information}
\def\semester{October, 2022}

\setlength{\oddsidemargin}{0.0 in}
\setlength{\evensidemargin}{0.0 in} 
\setlength{\topmargin}{-0.6 in} 
\setlength{\textwidth}{6.5 in} 
\setlength{\textheight}{8.5 in}
\setlength{\headsep}{0.75 in} 
\setlength{\parindent}{0 in}
\setlength{\parskip}{0.1 in}

\titleformat*{\section}{\Large\bfseries}


% prints box at top of first page with relevant info
\newcommand{\assignment}[3]{
   \pagestyle{myheadings}
   \thispagestyle{plain}
   \newpage
   \noindent
   \begin{center}
   \framebox{
      \vbox{\vspace{2mm}
    \hbox to 6.28in { {\bf \coursename
                        \hfill \semester} }
       \vspace{4mm}
       \hbox to 6.28in { {\Large \hfill #3  \hfill} }
       \vspace{2mm}
       \hbox to 6.28in { {\it #1 \hfill \texttt{#2}} }
      \vspace{2mm}}
   }
   \end{center}
   \vspace*{4mm}
}

\newcommand{\qsol}[1]{\section{#1}}
  % DO NOT CHANGE
\usepackage{graphicx,amssymb,amsmath,amsthm,mathrsfs}
\usepackage{multirow,makeidx,algorithmic,algorithm}
\usepackage{mathtools}
\usepackage{enumitem}
 

  % DO NOT CHANGE
% ADD YOUR CUSTOM NOTATION HERE

%%
% full alphabets of different styles
%%

% bf series
\def\bfA{\mathbf{A}}
\def\bfB{\mathbf{B}}
\def\bfC{\mathbf{C}}
\def\bfD{\mathbf{D}}
\def\bfE{\mathbf{E}}
\def\bfF{\mathbf{F}}
\def\bfG{\mathbf{G}}
\def\bfH{\mathbf{H}}
\def\bfI{\mathbf{I}}
\def\bfJ{\mathbf{J}}
\def\bfK{\mathbf{K}}
\def\bfL{\mathbf{L}}
\def\bfM{\mathbf{M}}
\def\bfN{\mathbf{N}}
\def\bfO{\mathbf{O}}
\def\bfP{\mathbf{P}}
\def\bfQ{\mathbf{Q}}
\def\bfR{\mathbf{R}}
\def\bfS{\mathbf{S}}
\def\bfT{\mathbf{T}}
\def\bfU{\mathbf{U}}
\def\bfV{\mathbf{V}}
\def\bfW{\mathbf{W}}
\def\bfX{\mathbf{X}}
\def\bfY{\mathbf{Y}}
\def\bfZ{\mathbf{Z}}

% bb series
\def\bbA{\mathbb{A}}
\def\bbB{\mathbb{B}}
\def\bbC{\mathbb{C}}
\def\bbD{\mathbb{D}}
\def\bbE{\mathbb{E}}
\def\bbF{\mathbb{F}}
\def\bbG{\mathbb{G}}
\def\bbH{\mathbb{H}}
\def\bbI{\mathbb{I}}
\def\bbJ{\mathbb{J}}
\def\bbK{\mathbb{K}}
\def\bbL{\mathbb{L}}
\def\bbM{\mathbb{M}}
\def\bbN{\mathbb{N}}
\def\bbO{\mathbb{O}}
\def\bbP{\mathbb{P}}
\def\bbQ{\mathbb{Q}}
\def\bbR{\mathbb{R}}
\def\bbS{\mathbb{S}}
\def\bbT{\mathbb{T}}
\def\bbU{\mathbb{U}}
\def\bbV{\mathbb{V}}
\def\bbW{\mathbb{W}}
\def\bbX{\mathbb{X}}
\def\bbY{\mathbb{Y}}
\def\bbZ{\mathbb{Z}}

% cal series
\def\calA{\mathcal{A}}
\def\calB{\mathcal{B}}
\def\calC{\mathcal{C}}
\def\calD{\mathcal{D}}
\def\calE{\mathcal{E}}
\def\calF{\mathcal{F}}
\def\calG{\mathcal{G}}
\def\calH{\mathcal{H}}
\def\calI{\mathcal{I}}
\def\calJ{\mathcal{J}}
\def\calK{\mathcal{K}}
\def\calL{\mathcal{L}}
\def\calM{\mathcal{M}}
\def\calN{\mathcal{N}}
\def\calO{\mathcal{O}}
\def\calP{\mathcal{P}}
\def\calQ{\mathcal{Q}}
\def\calR{\mathcal{R}}
\def\calS{\mathcal{S}}
\def\calT{\mathcal{T}}
\def\calU{\mathcal{U}}
\def\calV{\mathcal{V}}
\def\calW{\mathcal{W}}
\def\calX{\mathcal{X}}
\def\calY{\mathcal{Y}}
\def\calZ{\mathcal{Z}}


%%%%%%%%%%%%%%%%%%%%%%%%%%%%%%%%%%%%%%%%%%%%%%%%%%%%%%%%%%
% text short-cuts
\def\iid{i.i.d.\ } %i.i.d.
\def\ie{i.e.\ }
\def\eg{e.g.\ }
\def\Polya{P\'{o}lya\ }
%%%%%%%%%%%%%%%%%%%%%%%%%%%%%%%%%%%%%%%%%%%%%%%%%%%%%%%%%%

% set theory/measure theory
\def\collection{\calC}
\newcommand{\sigalg}[1]{\mathcal{#1}}
\def\borel{\calB} %Borel sets
\def\sigAlg{\sigalg{H}} %sigma-algebra
\def\filtration{\calF} %filtration
\newcommand{\msblSpace}[1]{(#1,\sigalg{#1})}
\newcommand{\measSpace}[2][\mu]{(#2,\sigalg{#2},#1)}
\newcommand{\borelSpace}[1]{(#1,\borel(#1))}
\newcommand{\measFuncs}[1]{\sigalg{#1}^f}
\newcommand{\pbblSpace}{(\Omega,\sigAlg)}
\newcommand{\probSpace}[1][\bbP]{(\Omega,\sigAlg,#1)}

\def\leb{\lambda}

\def\finv{f^{-1}} % inverse
\def\ginv{g^{-1}} % inverse

% group theory
\def\grp{\calG} %group

% operators
\def\P{\bbP} %fundamental probability
\def\E{\bbE} %expectation
% conditional expectation
\DeclarePairedDelimiterX\bigCond[2]{[}{]}{#1 \;\delimsize\vert\; #2}
\newcommand{\conditional}[3][]{\bbE_{#1}\bigCond*{#2}{#3}}
\def\Law{\mathcal{L}} %law; this is by convention in the literature
% \def\indicator{\mathds{1}} % indicator function
\def\1{{\mathbf 1}}
\def\indicator{\1}

% binary relations
\def\condind{{\perp\!\!\!\perp}} %independence/conditional independence
\def\equdist{\stackrel{\text{\rm\tiny d}}{=}} %equal in distribution
\def\equas{\stackrel{\text{\rm\tiny a.s.}}{=}} %euqal amost surely
\def\simiid{\sim_{\mbox{\tiny iid}}} %sampled i.i.d

% common vectors and matrices
\def\onevec{\mathbf{1}}
\def\iden{\mathbf{I}} % identity matrix
\def\supp{\text{\rm supp}}

% misc
% floor and ceiling
\DeclarePairedDelimiter{\ceilpair}{\lceil}{\rceil}
\DeclarePairedDelimiter{\floor}{\lfloor}{\rfloor}
\newcommand{\argdot}{{\,\vcenter{\hbox{\tiny$\bullet$}}\,}} %generic argument dot
\def\barX{\overline{X_n}}
%%%%%%%%%%%%%%%%%%%%%%%%%%%%%%%%%%%%%%%%%%%%%%%%%%%%%%%%%%


 
\graphicspath{{./figures/}}
\usepackage[hidelinks]{hyperref}
\usepackage[dvipsnames]{xcolor}
\usepackage[utf8]{inputenc}
\usepackage{amsmath}
\usepackage{bbm}
\usepackage{minted}

\def\bayesDCA{\href{https://giulianonetto.github.io/bayesdca/}{\textcolor{blue}{\{\mintinline{R}{bayesDCA}\}}} }

\begin{document}



\assignment{Giuliano Netto Flores Cruz}{cruz.gnf@gmail.com}{Description of models in bayesDCA}

\section*{Introduction}

In Decision Curve Analysis, we are typically interested in estimating the net benefit of adopting a given clinical decision strategy \cite{Vickers_2006}.
A decision strategy may be simply treating all patients under suspicion of a given condition, because the condition is
so deadly that any costs of intervention are outweighed -- e.g., treating aggressive cancer with surgery.
That is the ``Treat all" strategy, which alludes to intervening in all patients regardless of the nature of the intervention. Conversely, intervention may be risky and the underlying condition, 
not so much -- e.g., surgery for non-aggressive cancer.
In this case, a reasonable decision strategy is not to treat any patient -- the ``Treat none" strategy. A better decision strategy is, perhaps, 
to decide based on a predictive model which estimates a patient's likelihood of having aggressive disease. If the patient's likelihood
is above a decision threshold $t$, then we intervene. The same idea serves for binary tests, even though the test result is independent 
of our decision threshold $t$. We must then be able to estimate NB for these four types of decision strategies: treat all, treat none, model-based 
decisions, and decisions based on binary tests. 

The \bayesDCA R package helps with the above task while properly propagating uncertainty in the NB parameters. 
We can ultimately infer what's the best decision strategy, which strategies are useful at all, make pairwise comparisons between strategies, and 
estimate the expected cost of uncertainty imposed by the available data.
Here we briefly describe the models behind the Bayesian Decision Curve Analysis implemented in the 
\bayesDCA R package. We start with binary outcomes, which is the 
application \bayesDCA was primarily designed for. We then describe its extension to survival (time-to-event)
outcomes.

\section{Bayesian Decision Curve Analysis for binary outcomes}

Suppose we have a validation dataset with a binary outcome. For convenience, we refer to the outcome as disease status or diagnosis (diseased v.s. non-diseased), 
but the model generalizes for any binary endpoint (e.g., short-term death status or even disease subtype). For each patient, we also have 
the outcome prediction for models $m = 1,2,\dots, M$. This prediction can be a probability prediction (i.e., as in risk prediction models) or 
a binary prediction (i.e., as in binary tests). For a range of decision thresholds $t \in (0, 1)$, the Net Benefit for model $m$ at threshold $t$
is:
\begin{align}
    \textrm{NB}_{m, t} = \textrm{Se}_{m, t}\cdot p - (1 - \textrm{Sp}_{m, t})\cdot(1-p)\cdot \textrm{odds}(t) \label{nb-binary}
\end{align}
where $\textrm{odds}(t) = t/(1-t)$, $\textrm{Se}$ represents sensitivity, $\textrm{Sp}$ represents specificity, and $p$ represents the disease
prevalence or outcome proportion. The equation (\ref*{nb-binary})
has three unknown parameters, namely $\textrm{Se}_{m, t}$, $\textrm{Sp}_{m, t}$, and $p$. The first first two are model- and threshold-specific, while 
the $p$ is the same for all models and thresholds. To estimate these from cross-sectional data, the \bayesDCA R package implements the following model:
\begin{align}
    \textrm{Diseased} &\sim \textrm{Binomial}(\textrm{N} = n, p) \nonumber\\
    \textrm{True Positives}_{m, t} &\sim \textrm{Binomial}(\textrm{Diseased} = d, \textrm{Se}_{m, t}) \label{model-binary}\\
    \textrm{True Negatives}_{m, t} &\sim  \textrm{Binomial}(\textrm{Non-diseased} = n-d, \textrm{Sp}_{m, t}) \nonumber \\
    \nonumber\\
    p, \textrm{Se}_{m, t}, \textrm{Sp}_{m, t} &\sim \textrm{Beta}(\alpha_0, \beta_0) \label{prior-binary}
\end{align}
Together, model (\ref*{model-binary}) and prior (\ref*{prior-binary}) represent a series of beta-binomial models with closed-form solutions given by
\begin{align}
    p \ | \ \textrm{Diseased} = d, \textrm{N} = n &\sim \textrm{Beta}(\alpha_0 + d, n - d + \beta_0) \nonumber\\[5pt] 
    \textrm{Se}_{m, t} \ | \ \textrm{True Positives}_{m, t} = \textrm{tp}_{m, t}, \textrm{Diseased} = d  &\sim \textrm{Beta}(\alpha_0 + \textrm{tp}_{m, t}, d - \textrm{tp}_{m, t} + \beta_0) \label{post-binary}\\[5pt]
    \textrm{Sp}_{m, t} \ | \ \textrm{True Negatives}_{m, t} = \textrm{tn}_{m, t}, \textrm{Non-diseased} = n - d  &\sim \textrm{Beta}(\alpha_0 + \textrm{tn}_{m, t}, n - d - \textrm{tn}_{m, t} + \beta_0)\nonumber 
\end{align}
By default, \bayesDCA fixes the shared prior parameters to be $\alpha_0=\beta_0=1$, though the user may provide different values --
which may be model- and threshold-specific. For binary tests, where sensitivity and specificity are fixed for all thresholds, it may be beneficial
to elicitate more informative priors based on previous diagnostic test accuracy studies, for instance. However, prior elicitation in the case of risk prediction models
may be substantially more challenging due to the threshold-varying parameters. Since here the unknown quantities are relatively easy to estimate, using vague priors
is unlikely to be an issue.

We can jointly sample from the posterior distributions in (\ref*{post-binary}) to compute $\textrm{NB}_{m, t}$ as defined in equation (\ref*{nb-binary}). 
The \bayesDCA R package employs \textcolor{blue}{\href{https://mc-stan.org/}{Stan}} for this task, primarily because of \textcolor{blue}{\href{https://mc-stan.org/}{Stan}}'s 
rich model and sampling diagnostics. For instance, if the user specifies some really bad priors, \textcolor{blue}{\href{https://mc-stan.org/}{Stan}} may issue multiple warnings
about low effective (posterior) sample size and other convergence issues \cite{Carpenter2017}.

\subsection{Using external prevalence information for case-control data}

The model (\ref*{model-binary}) has a single parameter for the outcome proportion or disease prevalence, $p$. If we perform Decision Curve Analysis
in a case-control study, where the observed outcome proportion is determined by investigators, the observed data will not be suitable to estimate $p$.
Instead, we can use external prevalence information and estimate $p$. Its posterior distribution is then given by
\begin{align*}
    p \ | \ \textrm{Diseased} = d_{\textrm{ext.}}, \textrm{N} = n_{\textrm{ext.}} &\sim \textrm{Beta}(\alpha_0 + d_{\textrm{ext.}}, n_{\textrm{ext.}} - d_{\textrm{ext.}} + \beta_0)
\end{align*}
where $d_{\textrm{ext.}}$ and $n_{\textrm{ext.}}$ come from an external prevalence study (or any other source of prevalence information). This differs from common frequentist approaches that simply
plug-in a point estimate for $p$. Here, we propagate uncertainty around $p$ regardless of the source of prevalence information. The 
\bayesDCA R package allows the user to provide $d_{\textrm{ext.}}$ and $n_{\textrm{ext.}}$, while still using the observed data to 
estimate the sensitivity and specificity terms in equation (\ref*{nb-binary}).

\section{Bayesian Decision Curve Analysis for survival outcomes}

Decision Curve Analysis has been previously extended to survival (time-to-events) outcomes \cite{Vickers_2008}. Because of
right-censoring, we cannot condition on "having the event" or "not having the event". We then use Bayes theorem to rewrite the
Net Benefit as
\begin{align}
    \textrm{NB}^{\tau}_{m, t} = \left[1 - S\left(\tau \ | \ \widehat{r}_{m, \tau} > t \right) \right] \cdot \P\left[\hat{r}_{m, \tau} > t\right]  - S\left(\tau \ | \ \widehat{r}_{m, \tau} > t \right)  \cdot \P\left[\hat{r}_{m, \tau} > t\right] \cdot \textrm{odds}(t) \label{nb-survival}
\end{align}
where $\widehat{r}_{m, \tau}$ is the predicted risk of event from model $m$ at time horizon $\tau$ (e.g., the predicted one-year mortality according to model $m$),
$S\left(\tau \ | \ \widehat{r}_{m, \tau} > t \right)$ is the survival at time $\tau$ among patients with a positive prediction (i.e., for which $\widehat{r}_{m, \tau} > t$),
and $\P\left[\hat{r}_{m, \tau} > t\right]$ is the probability of a positive prediction. 

There are two unknown quantities in (\ref*{nb-survival}). The first one, $\P\left[\hat{r}_{m, \tau} > t\right]$, is estimated using another beta-binomial model:
\begin{align*}
    \textrm{Positives}_{m, t}^{\tau} \sim \textrm{Binomial}(N = n, \gamma_{m, \tau}^{(t)})
\end{align*}
where $\gamma_{m, \tau}^{(t)} = \P\left[\hat{r}_{m, \tau} > t\right]$. Notice that the outcome in this case is not censored -- we have predictions
for all $N=n$ patients from each model $m$. 

The second unknown in (\ref*{nb-survival}) is $S\left(\tau \ | \ \widehat{r}_{m, \tau} > t \right)$,
the survival among patients with a positive prediction (``positive patients"). 
In a frequentist framework, we could simply plug-in 
the Kaplan-Meier point estimates as previously suggested \cite{Vickers_2008}. However, in order to properly
 propagate uncertainty throughout equation (\ref*{nb-survival}),
we need a Bayesian model for $S\left(\tau \ | \ \widehat{r}_{m, \tau} > t \right)$.
In the following section, we explain such a model. We employ the Poisson trick for 
a piecewise-constant hazards model, in a similar approach to the one implemented in the
 \textcolor{blue}{\href{https://doi.org/10.18637/jss.v092.i09}{spBayesSurv}} R package \cite{Zhou_2020}.
The main difference is that, here, we don't have any covariates.

\subsection{Bayesian estimation of survival among positive patients}

To estimate $S\left(\tau \ | \ \widehat{r}_{m, \tau} > t \right)$, first consider partitioning time into $K$ intervals:
\begin{align}
    0 = \tau_0 < \tau_1 < \tau_2 < \cdots < \tau_{K-1} < \tau_K = \infty \label{time-partition}
\end{align}
Define $\lambda_k$ as the constant hazard in the $k^{th}$ interval given by $I_k = [\tau_{k-1}, \tau_{k})$, for $k=1, 2, \dots, K$.
Usually $\lambda_k$ is attached to a linear predictor term $\exp\left(x_i^T\beta\right)$ with characteristics $x_i$ from the $i^{th}$
patient (under a proportional hazards assumption). We don't need that term here since we are only interested in the baseline hazards.
 Then, the hazard for 
patient $i$ at time $t$ is modeled as:
\begin{align*}
    \lambda_i(t) = \lambda_k \ \ \ \ \textrm{for} \ \  t \in [\tau_{k-1}, \tau_k)
\end{align*}
Since (\ref{time-partition}) is a time partition the above can be further simplified to 
\begin{align}
    \lambda_{ik} = \lambda_k \label{pch}
\end{align}
Now, suppose we observe the exposure time $t_i$ and death indicator $d_i$ for patients $i=1,2,\dots, n$ (or similarly for any event other than death). The expression
(\ref*{pch}) only refers to patient- and interval-level information, so we may define corresponding pseudo-observations for each
individual at each time interval:
\begin{align}
    d_{ik} &= \mathbbm{1}\left\{t_i \in [\tau_{k-1}, \tau_k) \cap d_i = 1 \right\} = \begin{cases}
        1&\textrm{if patient $i$ died in the $k^{th}$ interval $I_k$}\\
        0&\textrm{if patient $i$ died in some other interval or was censored}
    \end{cases} \nonumber\\
    \label{pseudo-obs}\\
    t_{ik} &= \begin{cases}
        \tau_k - \tau_{k-1} &\textrm{if $t_i>\tau_k$}\\
        t_i - \tau_{k-1} &\textrm{if $t_i \in I_k$}\\
        0&\textrm{if $t_i < \tau_{k-1}$}
    \end{cases} = \textrm{exposure time of patient $i$ within $k^{th}$ interval}
    \nonumber
\end{align}
Notice that, if a patient $i$ died at time $t \in I_k$, then $d_{ik}=1$ and $d_{im}=0$ for all $m\neq k$. If a patient 
was censored, then $d_{ik}=0$ for all $k=1,2,\dots, K$. The Poisson trick comes from recognizing that we can treat 
the interval-specific death indicators $d_{ik}$ as if
\begin{align}
    d_{ik} \sim \textrm{Poisson}(t_{ik}\cdot \lambda_{ik}) \label{poisson-ik}
\end{align}
Since we are assuming independence across individuals $i=1,2,\dots, n$, we further simplify (\ref*{poisson-ik}) to:
\begin{align}
    d_k \sim \textrm{Poisson}(t_{k}\cdot \lambda_{k}) \label{poisson-k}
\end{align}
where
\begin{align*}
    t_k\cdot \lambda_k &= \sum_{i=1}^n t_{ik} \cdot \lambda_{ik} = \sum_{i=1}^n t_{ik} \cdot \lambda_{k} = \lambda_{k} \sum_{i=1}^n t_{ik} \\
    d_k &= \sum_{i=1}^n d_{ik} = \textrm{total deaths within $k^{th}$ interval} \\
    t_k &= \sum_{i=1}^n t_{ik} = \textrm{total exposure time within $k^{th}$ interval} 
\end{align*}
Placing a conjugate Gamma prior\footnote{$\alpha_k$ and $\beta_k$ are the shape and rate parameters, respectively.} $\Gamma(\alpha_k, \beta_k)$ for the parameters $\lambda_k$, we have the posterior distributions in closed form:
\begin{align}
    \lambda_k \ |\ d_k,t_k \ \sim \ \Gamma\left(\alpha_k + d_k, \beta_k + t_k\right) \label{post-hazards}
\end{align}
With draws from the posterior distributions for the hazard terms, we can then compute the cumulative hazard and survival probability at any time $t$ by:
\begin{align}
    S(t) &= \exp(-\Lambda(t)) \nonumber\\
    \Lambda(t) &= \sum_{k=1}^{K}\lambda_k \cdot T_k(t) \label{surv}\\
    T_k(t) &= \begin{cases}
        \tau_k - \tau_{k-1} &\textrm{if $t>\tau_k$}\\
        t - \tau_{k-1} &\textrm{if $t \in I_k$}\\
        0&\textrm{if $t < \tau_{k-1}$}\nonumber
    \end{cases}
\end{align}
where $T_k(t)$ is analogous to the interval-specific exposure time as defined in (\ref*{pseudo-obs}). 

Finally, to estimate the survival among positive patients, we fit the model defined in (\ref*{poisson-k} ---\ref*{surv}) using
data for patients whose model predictions $\widehat{r}_{m,\tau}$ fall above the decision threshold $t$.
\subsubsection{Choosing cutpoints and prior parameters}

%This is where your bibliography is generated. Make sure that your .bib file is actually called library.bib
\bibliography{library}

%This defines the bibliographies style. Search online for a list of available styles.
\bibliographystyle{ieeetr}

\end{document}
